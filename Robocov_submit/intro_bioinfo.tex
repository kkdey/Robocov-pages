\section{Introduction}

The gene expression data from nearly 50 tissues across more than 500 post-mortem donor individuals from Genotype Tissue Expression (GTEx) project has proved to be a valuable resource for understanding tissue-specific and tissue-shared genetic architecture~\cite{gtex2015, gtex2017, dey2017, aguet2019}. Here we are interested in one specific aspect of tissue-shared gene regulation: the correlation and partial correlation in gene expression for  different tissue pairs based on individual donor level data.  A major challenge in this context is the extensive amount of missing entries in gene expression data---each donor contributes only a subset of tissues for sequencing. 
Common imputation based methods do not work well here as reported in~\cite{dey2019}, owing to stringent assumptions about missing entries being close to some central tendency (median) or adhering to some low-dimensional representation of the observed entries~\cite{mazumder2010spectral,mazumder2015}. Popular shrinkage and/or sparse correlation or partial correlation estimators such as \textit{corpcor}~\cite{ledoit2003improved, schafer2005shrinkage}, GLASSO~\cite{friedman2008} or CLIME~\cite{cai2011} are not designed for data with missing values. 

A recently proposed approach, \CorShrink{}~\cite{dey2019}, co-authored by one of the authors of this paper (Dey), accounts for missing data through adaptive shrinkage~\cite{stephens2016} of correlations. \CorShrink{} does not guarantee a positive semidefinite (PSD) matrix as part of its EM-based framework, and necessitates a post-hoc modification to ensure a PSD correlation matrix. Furthermore, \CorShrink{} does not extend to conditional graph or partial correlation estimation. Here, we propose a new approach based on convex optimization: \Robocov{} that applies to both covariance and inverse covariance matrix estimation in the presence of missing data under the following regularization principles:
(a) the covariance matrix is sparse (i.e., has a few nonzero entries) or (b) the inverse covariance matrix is sparse. 

\Robocov{} does not \emph{impute} missing values per-se\footnote{Expectation Maximization (EM)~\cite{Dempster1977} methods typically used for estimation with missing values depend upon probabilistic modeling assumptions and lead to highly nonconvex problems
posing computational challenges.}---it directly estimates the covariance or inverse covariance matrices in the presence of missing values. To handle missing values, we consider a loss function that depends upon the pairwise covariance terms (computed based on the observed samples) but incorporates an adjustment to guard against our lack of knowledge regarding the missing observations. For inverse covariance estimation, \Robocov{} uses a robust optimization based approach~\cite{ben2009robust,bertsimas2011theory} that accounts for the uncertainty in estimating the pairwise sample covariance terms (due to the presence of missing values). Interestingly, both lead to convex optimization formulations that are amenable to modern optimization techniques~\cite{BV2004}---they are scalable to moderate-large scale instances; and unlike conventional EM methods (that lead to nonconvex optimization tasks), our estimators attain the global solution of the optimization formulations defining the 
\Robocov{} estimators. 

Our experiments suggest that \Robocov{} estimators for correlation and partial correlation matrices have lower false positive rate compared to competing approaches for missing data problems.  When applied to the GTEx gene expression data with~$\sim70\%$ missing data, \Robocov{} produced less cluttered and highly interpretable visualization of correlation and conditional graph architecture. From a biological perspective, a gene with high correlation in expression across many tissue pairs is potentially reflective of more systemic biological processes spanning multiple tissues. To this end, we prioritized genes based on the average \Robocov{} estimated correlation (partial correlation) across all tissue-pairs; we call them  \Robospan{} (\pRobospan{}) genes. A pathway enrichment analysis of  \Robospan{} (\pRobospan{}) genes showed enrichment in systemic functional pathways and the immune system.  SNPs linked to \Robospan{} (\pRobospan{}) genes were tested for autoimmune disease informativeness by applying Stratified LD-score regression (S-LDSC) to 11 common blood-related traits (5 autoimmune diseases and 6 blood cell traits; average $N$=306K), conditional on a broad set of  annotations. \Robospan{} and \pRobospan{} genes showed high disease informativeness for blood-related traits. In comparison, \Corspan{} genes defined similarly using the standard correlation estimator were non-informative. This highlights the biological and disease-level significance of our work.



