\section{Discussion}\label{sec:discussion}

Here we present \Robocov{}---a novel convex optimization-based framework for sparse estimation of covariance (correlation) and inverse covariance (partial correlation) matrix, given a data matrix with missing entries. Our approach does not rely on missing data imputation and hence mitigates the possible shortcomings of a sub-optimal imputation procedure (e.g., based on a low-rank assumption). Instead, \Robocov{} directly estimates the correlation or partial correlation matrix of interest via a regularized loss minimization framework. Although here we focus our analysis on gene expression analysis, \Robocov{} is a stand-alone generic tool that can be applied to any data with missing entries.
%when the data matrix contains missing entries and does not allow a low rank representation to enable accurate imputation. 

We have assessed the significance of our proposed \Robocov{} framework over standard methods from a methodological, biological and disease analysis perspective. \Robocov{} leads to sparse estimates and has a lower false positive rate compared to other competing methods. \Robocov{} estimator is visually less cluttered and captures more robust biological signal. In terms of disease informativeness, \Robospan{} and \pRobospan{} gene sets, generated from the \Robocov{} estimated correlation and partial correlation matrices, perform considerably better than the analogous \Corspan{} gene set defined from standard correlation estimator.  

%\Robocov{} also provides better disease signal than the standard approach --- \Robospan{} and \pRobospan{} gene sets are more informative for blood and autoimmune traits over \Corspan{} gene set. 

%Our method has several limitations that can motivate future research directions. First, \Robocov{} assumes that all missing entries in the data matrix are missing at random. 
There are several directions for future research. One such direction would be to incorporate covariate information underlying structured missing-ness to inform \Robocov{} estimators. For GTEx data, donor metadata such as cause of death, age, gender etc can serve as important covariates.  Second, we are interested in modifying \Robocov{} to learn shared correlation structure between gene expression and other genetic and epigenomic data such as transcript level expression, ATAc-seq data etc. Third, from application standpoint, \Robocov{} can also be used as an ingredient in item response models for large scale participant data that may contain extensive amount of missing entries, as in UK Biobank  \cite{Sulis2017, Bauermeister2019}.

\section*{URLs}
\small
\begin{itemize}
    \item \Robocov{} software \\
    \url{https://github.com/kkdey/Robocov}
    \item GTEx v6 data analysis, gene list,
    pathway enrichment results, gene sets,
    annotations\\
    \url{https://github.com/kkdey/Robocov-pages}
    \item Baseline-LD annotations:\\ 
    \url{https://data.broadinstitute.org/alkesgroup/LDSCORE/}
    \item Summary statistics:\\ 
    \url{https://data.broadinstitute.org/alkesgroup/sumstats_formatted/}
\end{itemize}

\normalsize
\section*{Acknowledgements}

We thank Alkes L. Price, Bryce van de Geijn and Rajarshi Mukherjee for helpful comments. Rahul Mazumder was partially supported by 
the Office of Naval Research ONR-N000141512342, ONR-N000141812298 (Young Investigator Award), the National Science Foundation (NSF-IIS-1718258) and IBM.


